\documentclass[a4paper]{ctexart}
\usepackage[dvipsnames]{xcolor}  % 扩展颜色支持
\usepackage{listings}  % 代码高亮宏包
\usepackage[hidelinks]{hyperref}

\usepackage[
    a4paper,       % 保持纸张类型不变
    left=2.5cm,       % 减小左边距
    right=2.5cm,      % 减小右边距
    top=2.5cm,      % 上边距
    bottom=2.5cm,   % 下边距
    includeheadfoot % 包含页眉页脚在边距内
]{geometry} % 引入页边距控制宏包

% 自定义C++高亮风格
\lstdefinestyle{MyCStyle}{
    % 基础样式
    language=C++,               % 使用C++语法规则(C语言兼容)
    basicstyle=\ttfamily\small, % 基础字体
    backgroundcolor=\color{gray!5}, % 背景色
    frame=single,                % 边框样式
    framesep=3pt,               % 边框内边距
    rulecolor=\color{black!30}, % 边框颜色
    % 行号设置
    numbers=left,               % 左侧行号
    numberstyle=\tiny\color{gray}, % 行号样式
    stepnumber=1,               % 每行显示行号
    numbersep=8pt,              % 行号与代码间距
    % 代码换行
    breaklines=true,            % 自动换行
    % breakatwhitespace=true,     % 只在空格处换行
    postbreak=\mbox{\textcolor{red}{$\hookrightarrow$}\space}, % 换行标记
    tabsize=4,                  % 制表符等效空格数
    showstringspaces=false,     % 字符串中不显示空格
    % 语法高亮颜色定义
    commentstyle=\color{ForestGreen}\itshape, % 注释样式
    keywordstyle=\color{blue}\bfseries,       % 关键字样式
    stringstyle=\color{red},                   % 字符串样式
    directivestyle=\color{purple}\bfseries,    % 预处理指令
    identifierstyle=\color{black},              % 标识符样式
    % 特殊字符高亮
    literate=*                  % 特殊字符处理
        {0}{{{\color{magenta}0}}}1
        {1}{{{\color{magenta}1}}}1
        {2}{{{\color{magenta}2}}}1
        {3}{{{\color{magenta}3}}}1
        {4}{{{\color{magenta}4}}}1
        {5}{{{\color{magenta}5}}}1
        {6}{{{\color{magenta}6}}}1
        {7}{{{\color{magenta}7}}}1
        {8}{{{\color{magenta}8}}}1
        {9}{{{\color{magenta}9}}}1
        {.0}{{{\color{magenta}.0}}}2
        {.1}{{{\color{magenta}.1}}}2
        {.2}{{{\color{magenta}.2}}}2
        {.3}{{{\color{magenta}.3}}}2
        {.4}{{{\color{magenta}.4}}}2
        {.5}{{{\color{magenta}.5}}}2
        {.6}{{{\color{magenta}.6}}}2
        {.7}{{{\color{magenta}.7}}}2
        {.8}{{{\color{magenta}.8}}}2
        {.9}{{{\color{magenta}.9}}}2
}

% C++关键字扩展 (包含C和C++常见关键字)
\lstset{style=MyCStyle,
    morekeywords={  % 添加额外关键字
        alignas, alignof, constexpr, decltype,
        noexcept, nullptr, static\_assert, thread\_local,
        override, final, using, dynamic\_cast,
        const\_cast, reinterpret\_cast, static\_cast,
        template, typename, namespace, explicit,
        inline, noexcept, constinit, consteval,
        concept, requires, co\_await, co\_return,
        co\_yield, char8\_t, char16\_t, char32\_t
    }
}

\title{C++ 基础算法示例}
\author{Haoming Bai}

\begin{document}
\maketitle
\tableofcontents

\section{概念}

Concept (概念) 是本 C++ 模板的一项核心概念. "概念"引进于 C++20, 是通过一些限定的组合, 进而约束模板的适用范围, 同时也给代码补全器提供便利. 模板还可以简化报错信息, 将抽象的重载决议简化为条件不满足. 例如在 "Addable" 这个概念出现后, 用户可以知道自己传入的参数类型不满足加法性质, 而不是奇怪地在层层编译器的堆栈展开之后, 发现自己的类型不能满足标准库中的一个奇怪函数. 本章节仅覆盖了最常用的一些基础概念, 一些不够常用的概念会分布在各个章节之中. 在本模板中, 除去图论, 红黑树部分由于编写时间较早或一些其它特殊原因, 并未使用概念以外, 所有代码都使用概念进行约束.

\lstinputlisting[language=C++, caption=concepts.cpp, style=MyCStyle]{./concepts.cpp}

\section{图论}

图是计算机数据结构中最复杂的结构之一. 图论 (Graph theory) 是数学的一个分支, 图是图论的主要研究对象. 图 (Graph) 是由若干给定的顶点及连接两顶点的边所构成的图形, 这种图形通常用来描述某些事物之间的某种特定关系. 顶点用于代表事物, 连接两顶点的边则用于表示两个事物间具有这种关系.

图分为有向图和无向图, 有向图意味着在一个图中, 所有的边是有方向的. 有向图就像漂流一样, 游客只能顺着水流或者水车, 从水位高的地方去往水位低的地方, 或者从水车从山脚到山顶. 也可以回忆一下高中学到的洋流图, 我至今仍然记得当年学到的, 在南极洲附近的那个西风漂流. 西风漂流应该是只能从西到东, 从东到西就要出事情. 当然, 就像漂流的游客可以从山顶漂到山脚, 也可以从山脚坐水车来到山顶, 在有向图中, 如果存在 A->B, 并不意味着不能存在 B->A. 当然无向图就要简单得多, 就像大多数的城市路网规划一样. 如果你走错了路, 本应该在顾戴路下高架却一路开到了漕宝路, 你也可以原路返回.

这份算法示例中的图全部是有向图, 一些强制需要无向图的算法都是用有向图去模拟的无向图. 因为起草的时间很早, 所以没有使用概念约束, 同时很多算法因为较为常见, 也没来得及编写注释, 还请读者谅解. 不过想到使用这份模板的人算法水平应该远高于我, 所以应该也不需要很详细的解释了.

\lstinputlisting[language=C++, caption=graph.cpp, style=MyCStyle]{./graph.cpp}

\section{数据结构}

数据结构是大学计算机专业的第一门专业课, 也是西北工业大学计算机基础教学中心的副主任姜学锋教授眼中对于算法能力帮助最大的一门课. 常见的在课上会提及的数据结构在 C++ 的标准库中通常都有, 或者存在更好的平替, 那么只要记住 ADT, 然后在编程中使用标准的实现就足够了. 这里关注的是那些不常见的.

\subsection{线段树}

线段树是算法竞赛中常用的用来维护区间信息的数据结构. 线段树可以在 $O(\log(N))$ 的时间复杂度内实现单点修改, 区间修改, 区间查询 (区间求和, 求区间最大值, 求区间最小值) 等操作. 线段树存在两种常见的存储方案. 一种是朴素地使用数组存储数据, 然后通过一系列的计算, 维护节点在树中的位置和下标的关系, 这种方法常数小, 但是写起来很复杂, 也不够灵活. 另外一种是使用 "动态开点" 的算法, 使用链式结构的二叉树, 同时使用线性的方法开辟内存空间, 这样就可以简化代码, 同时保证速度. 后一种方案可以把线段树写的很大, 同时可以支持持久化等高阶操作.

\subsubsection*{朴素线段树}

朴素线段树基于分治思想构建二叉树: 根节点覆盖整个区间, 每个非叶节点递归二分其区间 (左子树 $[l, mid]$, 右子树 $[mid+1, r]$) , 叶节点对应单元素区间. 节点存储区间聚合信息 (如和/最值) , 更新时自底向上回溯修正受影响的节点值, 查询时合并目标区间的分解子区间结果. 其优势在于: 
\begin{itemize}
	\item \textbf{高效查询/更新}: 单点更新与区间查询时间复杂度均为 $O(\log n)$
	\item \textbf{结构清晰}: 标准二叉树实现, 逻辑简单易理解
\end{itemize}
主要缺陷包括: 
\begin{itemize}
	\item \textbf{区间更新效率低}: 直接更新每个元素需 $O(n \log n)$ 时间
	\item \textbf{空间冗余}: 需 $4N$ 数组存储树结构, 空间复杂度 $O(n)$
\end{itemize}

尽管在正式文件中通常避免附着测试代码, 但为直观展示线段树通过运算符重载实现字符串加法、乘法及最值等聚合功能的灵活性, 此处特别提供了由AI编写的测试类作为补充示例. 

\lstinputlisting[language=C++, caption=segment\_tree/simple\_segtree.cpp, style=MyCStyle]{./segment_tree/simple_segtree.cpp}

\subsection*{可持久化线段树}

与之前介绍的普通线段树不同, 可持久化线段树 (persistent segment tree) 并非在原地修改节点, 而是通过“路径复制”创建一个新的版本, 旧版本仍可保存历史记录. 这意味着每次更新操作仅复制涉及的那条从根到叶的路径节点, 其他未改动的节点被共享使用, 因此可以高效维护多个历史状态, 而普通线段树更新会破坏原结构, 不支持版本回溯. 

可持久化线段树常被应用于 "区间第 $K$ 小" 问题:先对原数组进行离散化, 并为每个前缀 $A[1\dots i]$ 构建一个版本的值域线段树, 每个节点保存该值段在当前前缀下出现的次数. 查询$[l,r]$区间第 $K$ 小, 可视作 $CT[r]–CT[l–1]$ (两版本的减法) 后, 从根节点下行, 查找 $K$ 所在子区间, 从而在 $O(log(n))$ 内得出答案. 

至于“主席树” (Chair/Chairman Tree) 这一称呼, 据说其名字源于一位昵称带 "主席" 二字的选手推广此结构, 故后人暂借此称;也有人说是因为其推广者拼音缩写与中国某前国家主席相同, 因此得名. 无论来源如何, 这个名称在中文竞赛圈内被广泛接受, 几乎成了可持久化值域线段树的代名词. 

第 $K$ 小数的查询之所以能够高效实现, 关键在于版本树间共享大量节点, 仅针对一个插入或查询路径创建 $O(log n)$ 个新节点, 版本之间的差异小, 既保持了线段树查询的效率, 也支持历史版本的任意访问, 是处理静态区间 $K_{th}$ 问题的经典方案. 

\lstinputlisting[language=C++, caption=segment\_tree/persistent\_segtree.cpp, style=MyCStyle]{./segment_tree/persistent_segtree.cpp}

\subsection{并查集}

并查集 (Disjoint Set Union, DSU) 是一种高效处理集合合并与查询的数据结构, 常用于解决元素分组, 连通性判断等问题. 在算法竞赛中, 它被广泛应用于图论相关题目, 尤其是在处理无向图的连通分量, 最小生成树 (如 Kruskal 算法) 以及等价关系建模中. 例如, 给定若干合并操作与查询操作, 并查集可以在接近常数时间内判断两个元素是否属于同一集合, 极大提升算法效率. 此外, 在图像识别, 网络社群划分, 物理模拟等实际问题中, 并查集也同样展现了其强大的应用价值.

\lstinputlisting[language=C++, caption=dsu.cpp, style=MyCStyle]{./dsu.cpp}

\subsection{ST表}

ST表是一种很简单的数据结构, 在特定的题目里面比线段树的解法常数小, 因此还是应该掌握. ST表适用于这样一种场景: 可重复贡献的场景下的反复查询. ST表的查询非常快, 可以达到最坏 $O(1)$ 的量级, 但是ST表只能用于那种 $x*x=x$ 的特殊运算, 且不能进行修改.

\lstinputlisting[language=C++, caption=sparce\_table.cpp, style=MyCStyle]{./sparce_table.cpp}

\subsection{平衡搜索树}

\subsubsection*{C语言版本红黑树}

红黑树是一种高效的自平衡二叉搜索树. 它通过在普通二叉搜索树的基础上增加额外的信息 (节点颜色: 红或黑) 和遵循一组严格的规则 (红黑规则: 如根节点黑, 红色节点子必黑, 任一节点到叶子的路径包含相同数量黑节点等), 确保树在动态插入和删除操作后能保持相对平衡. 这种平衡性是其高效性能的关键.

这里提供了我早年编写的一个简易红黑树, 使用C语言和侵入式节点编写, 请慎用这份代码, 代码中可能有暗伤! 我已经让 Google Gemini 制作了一些测试和修复, 但是这棵树的稳定性依然堪忧! 等到我完成任何一种 C++ 代码实现的平衡搜索树(AVL或者红黑树)之后, 这份模板就会被删除.

此外, 这份代码中还包含了一些常见的二叉树算法, 包括但不限于节点序构造二叉树, 最近公共祖先等.

\lstinputlisting[language=C++, caption=rb\_tree\_c\_ver.c, style=MyCStyle]{./rb_tree_c_ver.c}

\section{散列方法}

如果数据的范围过大, 就需要一些快速方法缩小这个范围. 例如对于1000个自然数-字符串对, 完全可以使用数组存储, 使用下标代表key, 但是如果简单地使用数组, 一旦数据中出现一个1000000000000-"xyz", 那么内存占用就会过大. 这时候, 合适的散列方法就相当重要.

\lstinputlisting[language=C++, caption=hash\_methods.cpp, style=MyCStyle]{./hash_methods.cpp}

\section{计算几何}

图形学是计算机的一个重要分支. 在计算机图形学中, 我们常常需要处理图形的构造, 变换与交互, 例如判断两个物体是否相交, 生成可视化模型的边界, 或进行光线追踪渲染. 这些看似图形学的问题, 其背后往往依赖于严密的几何计算逻辑. 计算几何正是处理这类问题的基础学科, 它研究如何通过算法和数据结构来解决几何对象之间的关系与操作, 如求交, 最近点对, 凸包, 半平面交等问题.

计算几何不仅服务于图形学本身, 在机器人路径规划, 地图导航系统, CAD设计, 物理引擎甚至算法竞赛中也发挥着重要作用. 它为图形学提供了强有力的理论与算法支持, 使得我们能以更高效, 更精确的方式处理二维或三维空间中的复杂几何问题.

\lstinputlisting[language=C++, caption=geometry.cpp, style=MyCStyle]{./geometry.cpp}

\end{document}
